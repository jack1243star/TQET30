\documentclass[a4paper,11pt]{jctvcdoc}

\usepackage{geometry}[2010/02/12]

\usepackage{hyperref}
\hypersetup{colorlinks=true}
\usepackage{color,soul}

\usepackage[position=bottom]{subfig}
\captionsetup[subfloat]{position=top}
\usepackage{multirow}
\usepackage{dcolumn}
\newcolumntype{.}{D{.}{.}{-1}}
\usepackage{colortbl}
\usepackage{makecell}
\usepackage{longtable}
\usepackage{array}

\usepackage[strings]{underscore}
\usepackage{csquotes}
\MakeOuterQuote{"}
\EnableQuotes

\newcommand\None{}
\newcommand\NotSet{}
\makeatletter
\newcommand{\Option}[1]{\ifx\optOption\@empty\gdef\optOption{#1}\else\g@addto@macro\optOption{ \\ #1}\fi}
\newcommand{\ShortOption}[1]{\ifx\optShortOption\@empty\gdef\optShortOption{#1}\else\g@addto@macro\optShortOption{ \\ #1}\fi}
\newcommand{\Default}[1]{\ifx\optDefault\@empty\gdef\optDefault{#1}\else\g@addto@macro\optDefault{ \\ #1}\fi}
\newcommand{\clearOptions}{\gdef\optOption{}\gdef\optShortOption{}\gdef\optDefault{}}
\makeatother
\newenvironment{OptionTable}[1]{%
	\footnotesize
	\def\arraystretch{1.8}
	\clearOptions
	\begin{longtable}{l<{\makecell[tl]{\optOption}}%
	                  >{\texttt\bgroup}l<{\makecell[tl]{\optShortOption}\egroup}%
	                  c<{\makecell[tc]{\optDefault}}%
	                  >{\def\arraystretch{1.0}}p{0.5\textwidth}<{\clearOptions}}
	\caption{#1} \\
	\hspace*{12em}&&\hspace*{8em}&\kill
	\hline
	 \thead{Option} &
	 \egroup\thead{Shorthand}\bgroup &
	 \thead{Default} &
	 \thead{Description} \\
	\hline
	\endfirsthead
	\caption[]{#1 (Continued)} \\
	\hspace*{12em}&&\hspace*{8em}&\kill
	\hline
	 \thead{Option} &
	 \egroup\thead{Shorthand}\bgroup &
	 \thead{Default} &
	 \thead{Description} \\
	\hline
	\endhead
	 \multicolumn{4}{r}{Continued...}\\
	 \hline
	\endfoot
	 \hline
	\endlastfoot
}{%
	\hline
	\end{longtable}
}

\newenvironment{MacroTable}[1]{%
	\footnotesize
	\def\arraystretch{1.3}
	\clearOptions
	\begin{longtable}{lcp{0.5\textwidth}}
	 \caption{#1} \\
	%\hspace*{12em}&&\hspace*{8em}&\kill
	 \hline
	  \thead{Option} &
	  \thead{Default} &
	  \thead{Description} \\
	 \hline
	\endfirsthead
	 \caption[]{#1 (Continued)} \\
	 \hline
	  \thead{Option} &
	  \thead{Default} &
	  \thead{Description} \\
	 \hline
	\endhead
	 \multicolumn{3}{r}{Continued...}\\
	 \hline
	\endfoot
	 \hline
	\endlastfoot
}{%
	\end{longtable}
}

\title{HM Software Manual}
\author{%
	Frank Bossen
	\email{bossen@docomoinnovations.com}
	\and
	David Flynn
	\email{davidf@rd.bbc.co.uk}
	\and
	Karsten Sühring
	\email{Karsten.Suehring@hhi.fraunhofer.de}
}

%\jctvcmeeting{10th Meeting: Stockholm, SE, 11 -- 20 July 2012}
\jctvcdocnum{Software Manual}
\jctvcdocstatus{Input Document to JCT-VC}
\jctvcdocpurpose{Information}
\jctvcdocsource{AHG chairs}

\begin{document}
\maketitle
\begin{abstract}
This document is a user manual describing usage of reference software
for the HEVC project. It applies to the latest development version
of the software.
\end{abstract}

\tableofcontents
\listoftables

\section{General Information}
Reference software is being made available to provide a reference
implementation of the draft HEVC standard being developed by the Joint
Collaborative Team on Video Coding (JCT-VC) regrouping experts from
ITU-T SG 16 and ISO/IEC SC29 WG11. One of the main goals of the
reference software is to provide a basis upon which to conduct
experiments in order to determine which coding tools provide desired
coding performance. It is not meant to be a particularly efficient
implementation of anything, and one may notice its apparent
unsuitability for a particular use. It should not be construed to be a
reflection of how complex a production-quality implementation of a
future HEVC standard would be.

This document aims to provide guidance on the usage of the reference
software. It is widely suspected to be incomplete and suggestions for
improvements are welcome. Such suggestions and general inquiries may be
sent to the general JCT-VC email reflector on
\url{jct-vc@lists.rwth-aachen.de} (registration required).

\subsection*{Bug reporting}
Bugs should be reported on the issue tracker set up at
\url{http://hevc.kw.bbc.co.uk/trac/}

\section{Installation and compilation}
The software may be retrieved from one of the following SVN servers
(mirrored):
\begin{itemize}
\item \url{https://hevc.hhi.fraunhofer.de/svn/svn_HEVCSoftware/}
\item \url{svn://hevc.kw.bbc.co.uk/svn/jctvc-hm/}
\end{itemize}

Table~\ref{tab:project-files} enumerates various project files that are
provided for development environments.

\begin{table}[ht]
\footnotesize
\caption{Available project files}
\label{tab:project-files}
\centering
\begin{tabular}{ll}
\hline
 \thead{Environment} &
 \thead{Location of project file} \\
% Environment          & Location of project file \\
\hline
MS Visual Studio 8   & build/HM_vc8.sln \\
MS Visual Studio 9   & build/HM_vc9.sln \\
Xcode                & HM.xcodeproj \\
Linux                & build/linux/makefile \\
\hline
\end{tabular}
\end{table}

%%%%
%%%%
%%%%
\clearpage
\section{Using the encoder}
\begin{verbatim}
TAppEncoder 	[-h] [-c config.cfg] [--parameter=value]
\end{verbatim}

\begin{table}[ht]
\footnotesize
\centering
\begin{tabular}{lp{0.5\textwidth}}
\hline
 \thead{Option} &
 \thead{Description} \\
\hline
\texttt{-h} & Prints parameter usage. \\
\texttt{-c} & Defines configuration file to use.  Multiple configuration files
     may be used with repeated --c options. \\
\texttt{--}\emph{parameter}\texttt{=}\emph{value}
    & Assigns value to a given parameter as further described below.
      Some parameters are also supported by shorthand
      "--\em{opt}~\emph{value}".\\
\hline
\end{tabular}
\end{table}

Sample configuration files are provided in the cfg/ folder.

\subsection{GOP structure table}
Defines the cyclic GOP structure that will be used repeatedly
throughout the sequence. The table should contain GOPSize lines,
named Frame1, Frame2, etc. The frames are listed in decoding
order, so Frame1 is the first frame in decoding order, Frame2 is
the second and so on. Among other things, the table specifies all
reference pictures kept by the decoder for each frame. This
includes pictures that are used for reference for the current
picture as well as pictures that will be used for reference in
the future. The encoder will not automatically calculate what
pictures that has to be kept for future references, they have to
be specified. Note that some specified reference frames for
pictures encoded in the very first GOP after an IDR frame might
not be available. This is handled automatically by the encoder,
so the reference pictures can be given in the GOP structure table
as if there were infinitely many identical GOPs before the
current one. Each line in the table contains the parameters used
for the corresponding frame, separated by whitespace:

Type: Slice type, can be either I, P or B.

POC: Display order of the frame within a GOP, ranging from 1 to
GOPSize.

QPOffset: QP offset is added to the QP parameter to set the final
QP value to use for this frame.

QPFactor: Weight used during rate distortion optimization. Higher
values mean lower quality and less bits. Typical range is between
0.3 and 1.

temporal_id: Temporal layer of the frame. A frame cannot predict
from a frame with a higher temporal id. If a frame with higher
temporal IDs is listed among a frame's reference pictures, it is
not used, but is kept for possible use in future frames.

num_ref_pics_active: Size of reference picture lists L0 and L1,
indicating how many reference pictures in each direction that are
used during coding.

ref_pic: A value of 1 indicates that this is a reference picture,
0 that it is a non-reference picture.

num_ref_pics: The number of reference pictures kept for this frame.
This includes pictures that are used for reference for the
current picture as well as pictures that will be used for
reference in the future.

reference pictures: A space-separated list of num_ref_pics integers,
specifying the POC of the reference pictures kept, relative the
POC of the current frame. The picture list shall be ordered,
first with negative numbers from largest to smallest, followed by
positive numbers from smallest to largest (e.g. -1 -3 -5 1 3).
Note that any pictures not supplied in this list will be
discarded and therefore not available as reference pictures
later.

predict: Defines the value of the syntax element
inter_ref_pic_set_prediction_flag. A value of 0 indicates that
the reference picture set is encoded without inter RPS prediction
and the subsequent parameters deltaRIdx-1, deltaRPS, num_ref_idcs
and Reference_idcs are ignored and do not need to be present. A
value of 1 indicates that the reference picture set is encoded
with inter prediction RPS using the subsequent parameters deltaRIdx-1,
deltaRPS, num_ref_idcs and Reference_idcs in the line. A value of 2
indicates that the reference picture set is encoded with inter RPS but
only the deltaRIdx-1 parameters is needed. The deltaRPS, num_ref_idcs and
Reference_idcs values are automatically derived by the encoder based on
the POC and refPic values of the current line and the RPS pointed to by
the deltaRIdx-1 parameters.

deltaRIdx-1: The difference between the index of the curent RPS
and the predictor RPS minus 1.

deltaRPS: The difference between the POC of the predictor RPS and
POC the current RPS.

num_ref_idcs: The number of ref_idcs to encode for the current RPS.
The value is equal to the value of  num_ref_pics of the predictor
RPS plus 1.

reference idcs: A space-separated list of num_ref_idcs integers,
specifying the ref idcs of the inter RPS prediction. The value of
ref_idcs may be 0, 1 or 2 indicating that the reference picture
is a reference picture used by the current picture, a reference
picture used for future picture or not a reference picture
anymore, respectively. The first num_ref_pics of ref_idcs correspond
to the Reference pictures in the predictor RPS. The last ref_idcs
corresponds to the predictor picture.

For example, consider the following coding structure:

This coding structure is of size 4. The pictures are listed in
decoding order. Frame1 shall therefore describe picture with
POC=4. It references picture 0, and therefore has -4 as a
reference picture. Similarly, Frame2 has a POC of 2, and since it
references pictures 0 and 4, its reference pictures are listed as
"-2 2". Frame3 is a special case: even though it only references
pictures with POC 0 and 2, it also needs to include the picture
with POC 4, which must be kept in order to be used as a reference
picture in the future. The reference picture list for Frame3
therefore becomes "-1 1 3". Frame4 has a POC of 3 and its list of
reference pictures is "-1 1".

Inter RPS prediction may be used for Frame2, Frame3 and Frame4,
hence the predict parameter is set to 1 for these frames. Frame2
uses Frame1 as the predictor hence the deltaRIdx-1 is 0.
Similarly for Frame3 and Frame4 which use Frame2 and Frame3 as
predictors, respectively. The deltaRPS is equal to the POC of the
predictor minus the POC of the current picture, therefore the
deltaRPS for Frame2 is 4 -2 = 2, for Frame3 is 2 - 1 = 1 and for
Frame4 is 1 - 3 = -2.

In Frame2, reference pictures with POC 0 and 2 are used, so the
reference idcs for Frame2 are “1 1” indicating that the reference
picture “-4” in Frame1 is still a reference picture in Frame2 and
Frame1 is also a reference picture in Frame2. The reference idcs
for Frame3 are “1 1 1”. The first and second “1”s indicating that
the reference pictures “-2 2” in Frame2 are still reference
pictures in Frame3 and the last “1” indicating that Frame2 is
also a reference picture in Frame3. In Frame 4, the reference
idcs are ”0 1 1 0”. The first “0” indicates that the reference
pictures “-1” in Frame 3 is no longer a reference picture in
Frame4. The next two “1”s indicate that the reference pictures “1
3” are now reference pictures of Frame4. The final “0” indicates
that Frame3 is not a reference picture.

In order to specify this to the encoder, the parameters in
Table~\ref{tab:gop-example} could be used.

\begin{table}[ht]
\footnotesize
\caption{GOP structure example}
\label{tab:gop-example}
\centering
\begin{tabular}{lrrrr}
\hline
 \thead{} &
 \thead{Frame1} &
 \thead{Frame2} &
 \thead{Frame3} &
 \thead{Frame4} \\
\hline
Type                &   P  &    B   &      B   &       B \\
POC                 &   4  &    2   &      1   &       3 \\
QPoffset            &   1  &    2   &      3   &       3 \\
QPfactor            & 0.5  &  0.5   &    0.5   &     0.5 \\
temporal_id         &   0  &    1   &      2   &       2 \\
num_ref_pics_active &   1  &    1   &      1   &       1 \\
ref_pic             &   1  &    1   &      0   &       0 \\
num_ref_pics        &   1  &    2   &      3   &       2 \\
reference pictures  & $-$4 & $-$2 2 & $-$1 1 3 &  $-$1 1 \\
predict             &   0  &    1   &      1   &       1 \\
deltaRIdx$-$1       &      &    0   &      0   &       0 \\
deltaRPS            &      &    2   &      1   &    $-$2 \\
num_ref_idcs        &      &    2   &      3   &       4 \\
reference idcs      &      &  1 1   &  1 1 1   & 0 1 1 0 \\
\hline
\end{tabular}
\end{table}

Here, the frames used for prediction have been given higher
quality by assigning a lower QP offset. Also, the non-reference
frames have been marked as belonging to a higher temporal layer,
to make it possible to decode only every other frame. Note: each
line should contain information for one frame, so this
configuration would be specified as:

\begin{verbatim}
Frame1: P 4 1 0.5 0 1 1 1 -4 0
Frame2: B 2 2 0.5 1 1 1 2 -2 2 1 0 2 2 1 1
Frame3: B 1 3 0.5 2 1 0 3 -1 1 3 1 0 1 3 1 1 1
Frame4: B 3 3 0.5 2 1 0 2 -1 1 1 0 -2 4 0 1 1 0
\end{verbatim}

The values of deltaRIdx-1[, ]deltaRPS, num_ref_idcs and reference
idcs of FrameK can be derived from the POC value of FrameK and
the POC, num_ref_pics and reference pictures values of FrameM, where
K is the index of the RPS to be inter coded and the M is the
index of the reference RPS, as follows.

deltaRIdx[K] -1 = K - M - 1

deltaRPS[K] [ ]= POC[M] [ ]- POC[K][]

num_ref_idcs[K ]= num_ref_pics[M ]+ 1

for (j=0; j<= num_ref_pics[M]; j++)

{

if ((reference pictures[M][j] + deltaRPS[K]) == reference
pictures[K][i], 0 <= i < num_ref_idcs[K] ) [ ]{

[ ]reference idcs[K][j] = 1, if reference pictures[K][i] is used
by the current frame, or[]

[ ]reference idcs[K][j] [ ]= 2, if reference pictures[K][i] [ ]is
not used by the current frame.

}

else {

reference idcs[K][j] [ ]= 0[]

}

}

reference pictures[M][num_ref_pics[M]] is does not exist and is assumed to
be 0.

Note: The above (automatic) generation of the inter RPS parameter
values has been integrated into the encoder, and is activated by
the value of predict = 2 followed by the value of deltaRIdx-1,
only, as described above.



%%%%
%%%%
%%%%
\newgeometry{tmargin=1.6cm,lmargin=1cm,rmargin=1cm,bmargin=1in,nohead}
\subsection{Encoder parameters}

%%
%% File, I/O and source parameters
%%
\begin{OptionTable}{File, I/O and source parameters}
\Option{InputFile} &
\ShortOption{-i} &
\Default{\NotSet} &
Defines source YUV file.

Note: Raw image data is stored in YUV files in planar format. When the bit
depth of samples is larger than 8, each sample is encoded in 2 bytes (little
endian, LSB-justified).
\\

\Option{BistreamFile} &
\ShortOption{-b} &
\Default{\NotSet} &
Defines coded bit stream file.
\\

\Option{ReconFile} &
\ShortOption{-o} &
\Default{\NotSet} &
Defines reconstructed YUV file.
\\

\Option{SourceWidth} &
\ShortOption{-wdt} &
\Default{0} &
Defines width of source in luma samples.
\\

\Option{SourceHeight} &
\ShortOption{-hgt} &
\Default{0} &
Defines height of source in luma samples.
\\

\Option{InputBitDepth} &
\ShortOption{\None} &
\Default{8} &
Defines the bit depth of the source YUV file.
\\

\Option{InternalBitDepth} &
\ShortOption{\None} &
\Default{0 (InputBitDepth)} &
Defines the internal bit depth used for coding.
\\

\Option{OutputBitDepth} &
\ShortOption{\None} &
\Default{0 (InternalBitDepth)} &
Defines the bit depth of the reconstructed YUV file.
\\

\Option{CroppingMode} &
\ShortOption{\None} &
\Default{0} &
Defines the cropping/padding parameters to be applied to the source
file. The following modes are available:
\par
\begin{tabular}{cp{0.5\textwidth}}
0 & No cropping / padding \\
1 & Automatic padding to the next minimum CU size \\
2 & Padding according to parameters HorizontalPadding and VerticalPadding \\
3 & Cropping according to parameters CropLeft, CropRight, CropTop and
    CropBottom \\
\end{tabular}
\\

\Option{HorizontalPadding}%
\Option{VerticalPadding} &
\ShortOption{-pdx}%
\ShortOption{-pdy} &
\Default{0} &
Defines horizontal and vertical padding of source in luma samples.
Must be a multiple of the chroma resolution (e.g. a multiple of two for
4:2:0).
\\

\Option{CropLeft}%
\Option{CropRight}%
\Option{CropTop}%
\Option{CropBottom} &
\ShortOption{\None} &
\Default{0} &
Defines horizontal and vertical cropping of source in luma samples.
Must be a multiple of the chroma resolution (e.g. a multiple of two for
4:2:0).
\\

\Option{FrameRate} &
\ShortOption{-fr} &
\Default{0} &
Defines frame rate of source.
\\

\Option{FrameSkip} &
\ShortOption{-fs} &
\Default{0} &
Defines number of frames to skip at beginning of source.
\hl{Note: this feature is currently not functional.}
\\

\Option{FramesToBeEncoded} &
\ShortOption{-f} &
\Default{0 (all)} &
Defines number of frames to be encoded.
\\
\end{OptionTable}


%%
%% Unit definition parameters
%%
\begin{OptionTable}{Unit definition parameters}
\Option{MaxCUWidth} &
\ShortOption{\None} &
\Default{64} &
Defines maximum CU width.
\\

\Option{MaxCUHeight} &
\ShortOption{\None} &
\Default{64} &
Defines maximum CU height.
\\

\Option{MaxCUSize} &
\ShortOption{\None} &
\Default{64} &
Defines maximum CU size.
\\

\Option{MaxPartitionDepth} &
\ShortOption{-h} &
\Default{4} &
Defines the depth of the CU tree.
\\

\Option{QuadtreeTULog2MaxSize} &
\ShortOption{\None} &
\Default{6 ($= \mathrm{log}_2(64)$)} &
Maximum TU size in logarithm base 2.
\\

\Option{QuadtreeTULog2MinSize} &
\ShortOption{\None} &
\Default{2 ($= \mathrm{log}_2(4)$)} &
Minimum TU size in logarithm base 2.
\\

\Option{QuadtreeTUMaxDepthIntra} &
\ShortOption{\None} &
\Default{1} &
Defines depth of TU tree for intra CUs.
\\

\Option{QuadtreeTUMaxDepthInter} &
\ShortOption{\None} &
\Default{2} &
Defines depth of TU tree for inter CUs.
\\
\end{OptionTable}


%%
%% Coding structure parameters
%%
\begin{OptionTable}{Coding structure parameters}
\Option{IntraPeriod} &
\ShortOption{-ip} &
\Default{$-1$} &
Defines intra frame period.
A value of $-1$ implies an infinite period.
\\

\Option{DecodingRefreshType} &
\ShortOption{-dr} &
\Default{0} &
Defines the type of decoding refresh to apply for the picture at the
intra frame period.\par
\begin{tabular}{cp{0.5\textwidth}}
%\thead{Value} & \\
0 & applies an I picture (not a clean random access point). \\
1 & applies a non-IDR clean random access point (open GOP). \\
2 & applies an IDR random access point (closed GOP). \\
\end{tabular}
%A value of 0 applies an I picture (not a clean random access point).
%A value of 1 applies a non-IDR clean random access point (opened GOP).
%A value of 2 applies an IDR random access point (closed GOP).
\\

\Option{GOPSize} &
\ShortOption{-g} &
\Default{1} &
Defines size of picture hierarchy.
\\

\Option{Frame\emph{N}} &
\ShortOption{\None} &
\Default{\NotSet} &
Multiple options describing a table that defines the cyclic GOP
structure that will be used repeatedly throughout the equence.
The table should contain GOPSize elements.
\par
See section~\ref{sec:gop-structure} for further details.
\\

\Option{ListCombination} &
\ShortOption{-lc} &
\Default{true} &
Defines use of combined reference list for uni-prediction in B-slices.
\begin{itemize}
\item
  A value of 1 indicates that the combined reference list is derived
  from reference list~0 and reference list~1.
\item
  A value of 0 indicates that reference list~0 and reference list~1
  are identical and reference list~0 is used as the combined
  reference list.
\end{itemize}
NB: LComb can only be 0 in low delay coding (more precisely, when list 0
and list 1 are the same)
\\
\end{OptionTable}


%%
%% Motion estimation parameters
%%
\begin{OptionTable}{Motion estimation parameters}
\Option{FastSearch} &
\ShortOption{\None} &
\Default{true} &
Defines use of fast motion search. If this parameter has a value of 0,
full search method is used, otherwise, fast motion estimation method is
used.

Note: the current software allows the value to range from 0 to 2, but
only 0 (full) and 1 (fast) are supported.
\\

\Option{SearchRange} &
\ShortOption{-sr} &
\Default{96} &
Defines the search range used for motion estimation.

Note: the search range is defined around a predictor. Motion vectors
derived by the motion estimation may thus have values larger than the
search range.
\\

\Option{BipredSearchRange} &
\ShortOption{\None} &
\Default{4} &
Defines the search range used for bi-prediction refinement in the motion
estimation.
\\

\Option{HadamardME} &
\ShortOption{\None} &
\Default{true} &
Defines use of the Hadamard transform in fractional-pel motion
estimation. Otherwise, SAD is used instead.
\\

\Option{ASR} &
\ShortOption{\None} &
\Default{false} &
Defines use of adaptive search range, which dynamically adjusts the
motion search range according to the POC difference between the current
and the reference pictures.

SearchRagne’ = Round( SearchRange * ADAPT_SR_SCALE * abs( POCcur – POCref ) / RateGOPSize )
\\
\end{OptionTable}


%%
%% Mode decision parameters
%%
\begin{OptionTable}{Mode decision parameters}
\Option{LambdaModifier$N$} &
\ShortOption{-LM$N$} &
\Default{1.0} &
Specifies a value that is multiplied with the Lagrange multiplier
$\lambda$, for use in the rate-distortion optimised cost calculation
when encoding temporal layer~$N$.
\par
$N$ may be in the range 0--7.
\\

\Option{FEN} &
\ShortOption{\None} &
\Default{false} &
Defines use of fast encoder mode. Default is 0 (false).
When this option is specified, the following changes apply:
\begin{itemize}
\item In the SAD computation for blocks having size larger than 8, only
      the lines of even rows in the block are considered.
\item The number of iterations used in the bi-directional motion vector
      refinement in the motion estimation process is reduced from 4 to 1.
\end{itemize}
\\

\Option{FDM} &
\ShortOption{\None} &
\Default{true} &
Defines use of fast encoder decision for 2Nx2N merge mode. When this
option is specify the RD cost for the merge mode of the current
candidate is not evaluated if the merge skip mode was the best merge
mode for one of the previous candidates.
\\
\end{OptionTable}

%%
%% Quantization parameters
%%
\begin{OptionTable}{Quantization parameters}
\Option{QP} &
\ShortOption{-q} &
\Default{30.0} &
Defines base value of the quantization parameter.
\\

\Option{CbQpOffset}%
\Option{CrQpOffset} &
\ShortOption{-cbqpofs}%
\ShortOption{-crqpofs} &
\Default{0}%
\Default{0} &
Global offset to apply to luma QP to derive QP of Cb and Cr
respectively.  These options correspond to the values of cb_qp_offset
and cr_qp_offset, that are transmitted in the PPS.  Valid values are in
the range $[-12, 12]$.
\\

\Option{MaxCuDQPDepth} &
\ShortOption{\None} &
\Default{0} &
Defines max depth of a minimum CuDQP for sub-LCU-level delta QP.
MaxCuDQPDepth shall be greater than or equal to SliceGranularity.
\\

\Option{RDOQ} &
\ShortOption{\None} &
\Default{true} &
Defines whether rate-distortion-optimized quantization is used in the
encoder.
\\

\Option{DeltaQpRD} &
\ShortOption{-dqr} &
\Default{0} &
Defines maximum QP offset at slice level for multi-pass slice encoding.
In the encoder, each slice is tested multiple times by using slice QP
values in the range [-DeltaQpRD, DeptaQpRD], and the best QP value is
chosen as the slice QP.
\\

\Option{MaxDeltaQP} &
\ShortOption{-d} &
\Default{0} &
Defines maximum QP offset at largest coding unit level for block-level
adaptive QP assignment scheme. In the encoder, each largest coding unit
is tested multiple times by using the QP values in the range
[-MaxDeltaQP, MaxDeltaQP], and the best QP value is chosen as the QP
value of the largest coding unit.
\\

\Option{dQPFile} &
\ShortOption{-m} &
\Default{\NotSet} &
Defines file name containing list of QP value deltas. The n-th line
(where n is 0 for the first line) of this file corresponds to the QP
value delta for the picture with POC value n.
\\

\Option{AdaptiveQpSelection} &
\ShortOption{-aqps} &
\Default{false} &
Defines whether QP values for non-I frames will be calculated on the fly
based on statistics of previously coded frames.
\\
\end{OptionTable}



%%
%% Entropy coding parameters
%%
\begin{OptionTable}{Entropy coding parameters}
\Option{SBACRD} &
\ShortOption{\None} &
\Default{true} &
Defines use of bit counts from arithmetic coder in rate-distortion
decisions.
\\
\end{OptionTable}


%%
%% Slice coding parameters
%%
\begin{OptionTable}{Slice coding parameters}
\Option{SliceGranularity} &
\ShortOption{\None} &
\Default{0} &
Determines the depth in an LCU at which slices may begin and end. A
value of 0 means that slice addresses are LCU aligned. A value of 1
means that slices addresses may additionally be aligned with CUs that
has a width and height that is one half of an LCU. (depth of 1). Values
between 0 and 3 inclusive are permitted, provided that CUs at the given
depth are of size 16x16 or larger. For 64x64 LCUs, this means that the
maximum allowed SliceGranularity is 2 since that gives a CU of 16x16
which is the smallest allowed.
\\

\Option{SliceMode} &
\ShortOption{\None} &
\Default{0} &
Enables slice coding. When set to 1, the parameter SliceArgument will
specify the maximum number of CUs at a depth of SliceGranuarity in a
slice. When set to 2, the parameter SliceArgument will specify the
number of bytes in a slice.
\\

\Option{SliceArgument} &
\ShortOption{\None} &
\Default{\NotSet} &
Defines the maximum number of CUs or bytes in a slice depending on the
SliceMode setting.
\\

\Option{EntropySliceMode} &
\ShortOption{\None} &
\Default{0} &
Enables entropy slice coding. When set to 1, the parameter SliceArgument
will specify the maximum number of CUs in a slice. When set to 2, the
parameter SliceArgument will specify the number of bytes (LCEC) or bins
(CABAC) in a slice.
\\

\Option{EntropySliceArgument} &
\ShortOption{\None} &
\Default{\NotSet} &
Defines the maximum number of CUs or bytes/bins in an entropy slice
depending on the SliceMode setting.
\\

\Option{WaveFrontSynchro} &
\ShortOption{\None} &
\Default{false} &
Defines use of specific CABAC probabilities synchronization at the
beginning of each line of CTBs in order to produce a bitstream that can
be encoded or decoded using one or more cores.
\\
\end{OptionTable}



%%
%% Deblocking filter parameters
%%
\begin{OptionTable}{Deblocking filter parameters}
\Option{LoopFilterDisable} &
\ShortOption{\None} &
\Default{false} &
Flag to disable the loop deblocking filter.
\\

\Option{LFCrossSliceBoundaryFlag} &
\ShortOption{\None} &
\Default{true} &
Flag to enable in-loop filtering across slice boundaries. When it is set
to 1, in-loop filtering can cross slice boundaries. When it is set to 0,
in-loop filtering cannot cross slice boundaries.
\\

\Option{DeblockingFilterControlPresent}&
\ShortOption{\None}&
\Default{false}&
Enables or disables the presence of the deblocking filter control
parameters in the picture parameter set and in the slice header.
When disabled, the default deblocking filter parameters are used.
\\
\end{OptionTable}



%%
%% Coding tools parameters
%%
\begin{OptionTable}{Coding tools parameters}
\Option{ALF} &
\ShortOption{\None} &
\Default{true} &
Defines use of the adaptive loop filter.
\\

\Option{ALFLowLatencyEncode} &
\ShortOption{\None} &
\Default{false} &
Defines the operating mode (low latency or high efficiency) of the
adaptive loop filter.
\\

\Option{SAO} &
\ShortOption{\None} &
\Default{true} &
Defines use of sample adaptive offset (SAO).
\\

\Option{LMChroma} &
\ShortOption{\None} &
\Default{true} &
Defines use of the Intra_ChromaFromLuma prediction mode.
\\

\Option{NSQT} &
\ShortOption{\None} &
\Default{true} &
Enables the use of the non-square quadtree transform.
\\

\Option{ConstrainedIntraPred} &
\ShortOption{\None} &
\Default{false} &
When set to 1, only pixels from Intra blocks in the same slice as the
current block will be marked as available for Intra prediction.
\\

\Option{PCMEnabledFlag} &
\ShortOption{\None} &
\Default{false} &
Enables or disables the use of PCM.
\\

\Option{PCMLog2MaxSize} &
\ShortOption{\None} &
\Default{5 ($= \mathrm{log}_2(32)$)} &
Specifies log2 of the maximum PCM block size. When PCM is enabled, the
PCM mode is available for 2Nx2N intra PUs smaller than or equal to the
specified maximum PCM block size
\\

\Option{PCMLog2MinSize} &
\ShortOption{\None} &
\Default{3} &
Specifies log2 of the minimum PCM block size. When PCM is enabled, the
PCM mode is available for 2Nx2N intra PUs larger than or equal to the
specified minimum PCM block size.
\par
When larger than PCMLog2MaxSize, PCM mode is not used.
\\

\Option{PCMInputBitDepthFlag} &
\ShortOption{\None} &
\Default{1} &
If enabled specifies that PCM sample bit-depth is set equal to
InputBitDepth. Otherwise, it specifies that PCM sample bit-depth is set
equal to InternalBitDepth.
\\

\Option{PCMFilterDisableFlag} &
\ShortOption{\None} &
\Default{false} &
If enabled specifies that loop-filtering on reconstructed samples of PCM
blocks is skipped. Otherwise, it specifies that loop-filtering on
reconstructed samples of PCM blocks is not skipped.
% 0 = (loop-filtering is not skipped for PCM samples).
\\

\Option{weighted_pred_flag} &
\ShortOption{-wpP} &
\Default{false} &
Controls the use of weighted prediction in P slices.
\\

\Option{weighted_bipred_idc} &
\ShortOption{-wpB} &
\Default{0} &
Controls the use of weighted prediction in B slices.
\par
\begin{tabular}{cp{0.5\textwidth}}
0 & Disabled \\
1 & Explicit weighted prediction \\
2 & Implicit weighted prediction \\
\end{tabular}
\\

\Option{SignHideFlag} &
\ShortOption{-SBH} &
\Default{true} &
If enabled specifies that for each 4x4 coefficient group for which the
number of coefficients between the first nonzero coefficient and the
last nonzero coefficient along the scanning line exceeds 4, the sign bit
of the first nonzero coefficient will not be directly transmitted in the
bitstream, but may be inferred from the parity of the sum of all nonzero
coefficients in the current coefficient group.
\\

\Option{TMVPMode} &
\ShortOption{\None} &
\Default{1} &
Controls the temporal motion vector prediction mode.
\par
\begin{tabular}{cp{0.5\textwidth}}
  0 & Disabled for all slices. \\
  1 & Enabled for all slices. \\
  2 & Disabled only for the first picture of each GOPSize. \\
\end{tabular}
\\

\Option{TS} &
\ShortOption{\None} &
\Default{false} &
Enables or disables transform-skipping mode decision for 4x4 TUs
\footnote{Enables transform_skip_enabled and per 4x4 TU tests}.
\\

\Option{TSFast} &
\ShortOption{\None} &
\Default{false} &
Enables or disables reduced testing of the transform-skipping mode
decision for chroma TUs.  When enabled, no RDO search is performed for
chroma TUs, instead they are transform-skipped if the four corresponding
luma TUs are also skipped.
\par
This option has no effect if TS is disabled.
\\
\end{OptionTable}



%%
%% Miscellaneous parameters
%%
\begin{OptionTable}{Miscellaneous parameters}
\Option{SEIPictureDigest} &
\ShortOption{\None} &
\Default{true} &
Enable or disable the calculation and insertion of the picture_digest
SEI messages.  When this parameter is set to 0, the feature is disabled.
When set to 1 (default), the feature is enabled and the encoder will
write the MD5 of each locally decoded picture to the log using the
format: [MD5:d41d8cd98f00b204e9800998ecf8427e].
\\
\end{OptionTable}

%%
%%
%%
\subsection{Hardcoded encoder parameters}
\subsubsection*{CommonDef.h}
ADAPT_SR_SCALE
Defines a scaling factor used to derive the motion search range is
adaptive (see ASR configuration parameter). Default value is 1.

\subsubsection*{TypeDef.h}
TBD

\subsubsection*{TAppEncCfg.cpp}
MAX_PAD_SIZE
Defines the multiple to which the size of a source picture is padded
when using the PAD configuration parameter. Default value is 16.


%%
%%
%%
\clearpage
\section{Using the decoder}
\begin{verbatim}
TappDecoder -b str.bin -o dec.yuv [options]
\end{verbatim}

\begin{OptionTable}{Decoder options}
\Option{} &
\ShortOption{-h} &
\Default{\None} &
Prints usage information.
\\

\Option{} &
\ShortOption{-o} &
\Default{\NotSet} &
Defines reconstructed YUV file name.
\\

\Option{} &
\ShortOption{-s} &
\Default{0} &
Defines the number of pictures in decoding order to skip.
\\

\Option{} &
\ShortOption{-d} &
\Default{0 (Native)} &
Defines the bit depth of the reconstructed YUV file (a value 0 indicates
that the native bit depth is used)
\\

\Option{SEIPictureDigest} &
\ShortOption{\None} &
\Default{1} &
Enable or disable verification of any picture_digest SEI messages. When
this parameter is set to 0, the feature is disabled and all messages are
ignored. When set to 1 (default), the feature is enabled and the decoder
has the following behaviour:
\begin{itemize}
\item
  For each decoded picture, the MD5 is calculated and written to the
  log.  If the input bitstream has a picture_digest SEI message for the
  corresponding decoded picture, the two MD5s are compared. Decoding
  will continue even if there is a mismatch.

\item
  After decoding is complete, if any MD5sum comparison failed, a warning
  is printed and the decoder exits with the status EXIT_FAILURE

\item
  The per-picture MD5 log message has the following formats:
  [MD5:d41d8cd98f00b204e9800998ecf8427e,(OK)],
  [MD5:d41d8cd98f00b204e9800998ecf8427e,(unk)],
  [MD5:d41d8cd98f00b204e9800998ecf8427e,(***ERROR***)] [rxMD5:b9e1...]
  where, "(unk)" implies that no MD5 was signalled for this picture,
  "(OK)" implies that the decoder agrees with the signalled MD5,
  "(***ERROR***)" implies that the decoder disagrees with the signalled
  MD5. "[rxMD5:...]" is the signalled MD5 if different.
\end{itemize}
\\
\end{OptionTable}


\end{document}
